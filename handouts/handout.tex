\documentclass[12pt]{scrartcl}


\setlength{\parindent}{0pt}
\setlength{\parskip}{.25cm}

\usepackage{graphicx}

\usepackage{xcolor}

\definecolor{darkred}{rgb}{0.5,0,0}
\definecolor{darkgreen}{rgb}{0,0.5,0}
\usepackage{hyperref}
\hypersetup{
  letterpaper,
  colorlinks,
  linkcolor=red,
  citecolor=darkgreen,
  menucolor=darkred,
  urlcolor=blue,
  pdfpagemode=none,
  pdftitle={CSCE 156 Lab Handout},
  pdfauthor={Christopher M. Bourke},
  pdfsubject={},
  pdfkeywords={}
}

\definecolor{MyDarkBlue}{rgb}{0,0.08,0.45}
\definecolor{MyDarkRed}{rgb}{0.45,0.08,0}
\definecolor{MyDarkGreen}{rgb}{0.08,0.45,0.08}

\definecolor{mintedBackground}{rgb}{0.95,0.95,0.95}
\definecolor{mintedInlineBackground}{rgb}{.90,.90,1}

%\usepackage{newfloat}
\usepackage[newfloat=true]{minted}
\setminted{mathescape,
               linenos,
               autogobble,
               frame=none,
               framesep=2mm,
               framerule=0.4pt,
               %label=foo,
               xleftmargin=2em,
               xrightmargin=0em,
               startinline=true,  %PHP only, allow it to omit the PHP Tags *** with this option, variables using dollar sign in comments are treated as latex math
               numbersep=10pt, %gap between line numbers and start of line
               style=default, %syntax highlighting style, default is "default"
               			    %gallery: http://help.farbox.com/pygments.html
			    	    %list available: pygmentize -L styles
               bgcolor=mintedBackground} %prevents breaking across pages
               
\setmintedinline{bgcolor={mintedBackground}}
\setminted[text]{bgcolor={mintedBackground},linenos=false,autogobble,xleftmargin=1em}
%\setminted[php]{bgcolor=mintedBackgroundPHP} %startinline=True}
\SetupFloatingEnvironment{listing}{name=Code Sample}
\SetupFloatingEnvironment{listing}{listname=List of Code Samples}


\title{CSCE 156 -- Computer Science II}
\subtitle{Lab 10.0 - JDBC in a Webapp II}
\author{~}
\date{~}

\begin{document}

\maketitle

\section*{Prior to Lab}

\begin{enumerate}
  \item Review this laboratory handout prior to lab.
  \item Make sure that the Albums database is installed and available 
  	in your MySQL instance on CSE
\end{enumerate}

\section*{Lab Objectives \& Topics}
Following the lab, you should be able to:
\begin{itemize}
  \item Write SQL queries to insert into a database using JDBC
  \item Handle data integrity problems arising from bad user input 
    and database integrity constraints (foreign key constraints)
  \item Have further exposure to a multi-tiered application using 
    Java Servlets and the HTTP request/response protocol.
\end{itemize}


\section*{Peer Programming Pair-Up}

To encourage collaboration and a team environment, labs will be
structured in a \emph{pair programming} setup.  At the start of
each lab, you will be randomly paired up with another student 
(conflicts such as absences will be dealt with by the lab instructor).
One of you will be designated the \emph{driver} and the other
the \emph{navigator}.  

The navigator will be responsible for reading the instructions and
telling the driver what to do next.  The driver will be in charge of the
keyboard and workstation.  Both driver and navigator are responsible
for suggesting fixes and solutions together.  Neither the navigator
nor the driver is ``in charge.''  Beyond your immediate pairing, you
are encouraged to help and interact and with other pairs in the lab.

Each week you should alternate: if you were a driver last week, 
be a navigator next, etc.  Resolve any issues (you were both drivers
last week) within your pair.  Ask the lab instructor to resolve issues
only when you cannot come to a consensus.  

Because of the peer programming setup of labs, it is absolutely 
essential that you complete any pre-lab activities and familiarize
yourself with the handouts prior to coming to lab.  Failure to do
so will negatively impact your ability to collaborate and work with 
others which may mean that you will not be able to complete the
lab.  

\section*{JDBC in a Web Application II}

In this lab you will continue to work with the Java Database Connectivity 
API (JDBC)  by finishing some new functionality to the Album web application 
from the prior lab.  If you need to, you can check out the project for this
lab from GitHub at \url{https://github.com/cbourke/CSCE156-Lab10}.

In this lab you will add functionality to the ``Add Album'' button/page
to allow a user to enter an album by providing the title, band (by name), 
year, and album number.  Again, the focus will be on the JDBC functionality 
for inserting these records, but you are invited to look into how the HTML 
web form posts to an XML configured (web.xml) Java Servlet 
(\mintinline{java}{AlbumAdderServlet.java}) which hands over the logic 
of inserting the album to the database using another Plain Old Java 
Object (POJO), \mintinline{java}{AlbumAdder.java}.

Your task is to make changes to the \mintinline{java}{AlbumAdder} class 
to take the four pieces of information provided by the user and insert 
records in the Album database (specifically the Albums and the Bands 
tables).  Most of the code has been provided, but you will need to 
implement the \mintinline{java}{addAlbumToDatabase()} method as specified
in the JavaDocs.  Keep the following in mind.

\begin{itemize}
  \item The Servlet provides all four arguments as strings.  If the
  	user does not enter valid data (non-integer values for year or
	album or empty strings), then you should handle those situations
	appropriately.
  \item The user should not be expected to know the internal keys to 
    a database.  Thus the page only asks them to provide a band name.  
    To preserve the integrity of data, you should not insert duplicate 
    band records.  Therefore, you should check to see if the Band record 
    already exists (according to its name).  If it does, use that foreign 
    key; if it doesn't, then you'll also need to insert the Band record.
  \item You cannot insert an album record into the Albums table 
    without a valid band ID.  This is because the Albums table has a 
    foreign key reference to a band.  This is good database design it 
    ensures good data integrity.  Unfortunately, it presents a problem: 
    we can insert a band record, but we need to immediately pull the 
    auto-generated key of the inserted record so we can use it in the 
    album record.  There are many mechanisms for dealing with this, 
    unfortunately a lot of them are vendor-specific (they work with one 
    database, but not others).
    
    One mechanism in mysql is the \mintinline{sql}{LAST_INSERT_ID()} 
    function which returns the ID of the last inserted record (within the
    session/connection) with an auto generated key.  You can query the 
    database using JDBC to call this function as follows.
\end{itemize}
    
\begin{minted}{java}
//after you have inserted a record...
PreparedStatement ps;
ps = conn.prepareStatement("SELECT LAST_INSERT_ID()");
ResultSet rs = ps.executeQuery();
//increment to first (and only) record:
rs.next();
int foreignKeyId = rs.getInt("LAST_INSERT_ID()");
\end{minted}

\subsection*{Deploying Your App}

Refer to the instructions in the previous lab for how to 
deploy your application.  Recall that
you need to use the JEE version of Eclipse and make changes
to the \mintinline{java}{unl.cse.albums.DatabaseInfo} file
to use your MySQL credentials.

\section*{Advanced Activities (Optional)}

\begin{enumerate}
  \item Every piece of data coming from a webform is represented with
    a \mintinline{java}{String}. The \mintinline{java}{AlbumAdderServlet} 
    had to do the 
    necessary conversion and handle the situation where invalid 
    (non-numeric) data was provided.   We could have also done the 
    validation at the client tier to prevent bad data from being 
    sent to the server.  Add some Javascript to your page to validate 
    the data to prevent the user from clicking the ``Add'' button 
    unless the year and album number are valid.  
  \item A Cross-Site Scripting (XSS) attack is an attack that is 
    similar to an SQL injection attack that involves an attacker who 
    injects HTML into a webpage by submitting it as legitimate input 
    but the application fails to properly \emph{sanitize} the HTML 
    when it gets served back to the user.  We will demonstrate that 
    our webapp is vulnerable to this attack by hacking it.
    \begin{enumerate}
      \item Modify your Album database and extend the length of the 
      \mintinline{sql}{Band.BandName} column by opening your mysql 
      client and executing the following SQL:
      
      \begin{minted}{sql}
      alter table Band change column name
        name varchar(2048) not null default '';
      \end{minted}
      
      \item Navigate to the page where you enter albums/bands and enter 
      the following data:
      \begin{itemize}
        \item Album Title: \emph{Yeah Ghost}
        \item Band Name (exactly as shown, but no line breaks): 

\begin{minted}{text}
Zero 7</a></td><td>2009</td></tr></table><p>For more, 
click <a href=
"http://cse.unl.edu/~cbourke/cse156/passwordStealer.html">
HERE</a><!--
\end{minted}

        \item Album Year: 2009
	    \item Album Number: 4
      \end{itemize}
      \item Submit the new Album and view the list--what has happened?  
        Click the new link at the bottom and see what happens.
	  \item This attack was not that sophisticated.  Imagine injecting
        JavaScript which operates in silence and in the background and 
        sends all browser data to another site and you can get an idea 
        of how severe the problem can be.
      \item Read more on how to prevent these kinds of attacks:\\
        \url{http://en.wikipedia.org/wiki/Cross-site_scripting} \\
	    \url{https://www.owasp.org/index.php/Cross-site_Scripting_(XSS)} \\
        Modify your project to protect against these sort of attacks.
    \end{enumerate}
  \item Make this application more complete by adding more functionality
  	so that users can insert songs, bands, etc.\ as well as delete 
	records from the database.

\end{enumerate}

\end{document}
